%%%%%%%%%%%%%%%%%%%%%%%%%%%%%%%%%%%%%%%%%
% Beamer Presentation
% LaTeX Template
% Version 1.0 (10/11/12)
%
% This template has been downloaded from:
% http://www.LaTeXTemplates.com
%
% License:
% CC BY-NC-SA 3.0 (http://creativecommons.org/licenses/by-nc-sa/3.0/)
%
%%%%%%%%%%%%%%%%%%%%%%%%%%%%%%%%%%%%%%%%%

%----------------------------------------------------------------------------------------
%	PACKAGES AND THEMES
%----------------------------------------------------------------------------------------



\documentclass{beamer}


\definecolor{dcorange}{HTML}{F05A28}
\setbeamercolor{structure}{bg=black, fg=dcorange}


\usetheme{Warsaw}

\usepackage{graphicx} % Allows including images
\usepackage{booktabs} % Allows the use of \toprule, \midrule and \bottomrule in tables
\usepackage{hyperref} % Allows the use of \toprule, \midrule and \bottomrule in tables
\usepackage{textpos}

%----------------------------------------------------------------------------------------
%	TITLE PAGE
%----------------------------------------------------------------------------------------

\titlegraphic{\includegraphics[width=.6\textwidth,height=.3\textheight]{images/dc_logo.jpg}}
\title[Introduction to PySemantic]{Introduction to PySemantic}
% The short title appears at the bottom of every slide, the full title is only on the title page

%\titlegraphic{\includegraphics[width=.5\textwidth,height=.5\textheight]{dc_logo.jpg}}
\author{Jaidev Deshpande} % Your name
\institute[DataCulture Analytics Company] % Your institution as it will appear on the bottom of every slide, may be shorthand to save space


%{
%DataCulture Analytics Company \\ % Your institution for the title page
%%\includegraphics[width=2cm]{dc_logo.jpg}
%\medskip
%\textit{jaidev@dataculture.in} % Your email address
%}
\date{\today} % Date, can be changed to a custom date

%\titlegraphic{\includegraphics[width=2cm]{dc_logo.jpg}}

%\addtobeamertemplate{frametitle}{}{%
%    \begin{textblock*}{100mm}(.85\textwidth,-1cm)
%        \includegraphics[height=1cm,width=2cm]{dc_logo.jpg}
%\end{textblock*}}



\begin{document}

\begin{frame}
\titlepage % Print the title page as the first slide
\end{frame}

\begin{frame}
\frametitle{Motivation}
\begin{itemize}
\item Typical data analysis pipeline:\\
Data Ingest $\rightarrow$ Exploratory Analysis $\rightarrow$ Feature Engineering $\rightarrow$ Machine Learning $\rightarrow$ Insights!
\item Data scientists often work in large teams.
\item Communication about data ingest is important.
\item Messy data $\Rightarrow$ more communication.
\end{itemize}
\end{frame}

\begin{frame}
    \frametitle{Why PySemantic?}
    \begin{itemize}
        \item Problem: How do I effectively communicate about data?
        \item Existing solutions:\\
            \begin{enumerate}
                \item Text documentation
                \item Ad-hoc scripts to clean or validate the data
                \item Version control
            \end{enumerate}
        \item Don't scale with the diversity of the data.
        \item The process is \textit{reactive}
        \item The process is unnecessarily redundant.
    \end{itemize}
\end{frame}

\begin{frame}
    \frametitle{Why PySemantic?}
    \begin{itemize}
        \item Group all datasets under \textit{projects}.
        \item A centralized data dictionary that holds properties of all
            datasets under a project.
        \item A single entry point into the software framework required for
            reading, cleaning and validating a dataset.
        \item Reproducibility across teams and individuals.
    \end{itemize}
\end{frame}

\begin{frame}
    \frametitle{Getting Started}
    \url{https://github.com/motherbox/pysemantic}
\end{frame}


%----------------------------------------------------------------------------------------

\end{document} 
