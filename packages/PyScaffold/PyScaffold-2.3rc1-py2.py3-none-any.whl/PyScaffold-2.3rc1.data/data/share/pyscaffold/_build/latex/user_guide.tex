% Generated by Sphinx.
\def\sphinxdocclass{report}
\documentclass[letterpaper,10pt,english]{sphinxmanual}
\usepackage[utf8]{inputenc}
\DeclareUnicodeCharacter{00A0}{\nobreakspace}
\usepackage{cmap}
\usepackage[T1]{fontenc}
\usepackage{babel}
\usepackage{times}
\usepackage[Bjarne]{fncychap}
\usepackage{longtable}
\usepackage{sphinx}
\usepackage{multirow}


\title{pyscaffold Documentation}
\date{July 29, 2014}
\release{0.9.post3}
\author{Blue Yonder}
\newcommand{\sphinxlogo}{}
\renewcommand{\releasename}{Release}
\makeindex

\makeatletter
\def\PYG@reset{\let\PYG@it=\relax \let\PYG@bf=\relax%
    \let\PYG@ul=\relax \let\PYG@tc=\relax%
    \let\PYG@bc=\relax \let\PYG@ff=\relax}
\def\PYG@tok#1{\csname PYG@tok@#1\endcsname}
\def\PYG@toks#1+{\ifx\relax#1\empty\else%
    \PYG@tok{#1}\expandafter\PYG@toks\fi}
\def\PYG@do#1{\PYG@bc{\PYG@tc{\PYG@ul{%
    \PYG@it{\PYG@bf{\PYG@ff{#1}}}}}}}
\def\PYG#1#2{\PYG@reset\PYG@toks#1+\relax+\PYG@do{#2}}

\expandafter\def\csname PYG@tok@gd\endcsname{\def\PYG@tc##1{\textcolor[rgb]{0.63,0.00,0.00}{##1}}}
\expandafter\def\csname PYG@tok@gu\endcsname{\let\PYG@bf=\textbf\def\PYG@tc##1{\textcolor[rgb]{0.50,0.00,0.50}{##1}}}
\expandafter\def\csname PYG@tok@gt\endcsname{\def\PYG@tc##1{\textcolor[rgb]{0.00,0.27,0.87}{##1}}}
\expandafter\def\csname PYG@tok@gs\endcsname{\let\PYG@bf=\textbf}
\expandafter\def\csname PYG@tok@gr\endcsname{\def\PYG@tc##1{\textcolor[rgb]{1.00,0.00,0.00}{##1}}}
\expandafter\def\csname PYG@tok@cm\endcsname{\let\PYG@it=\textit\def\PYG@tc##1{\textcolor[rgb]{0.25,0.50,0.56}{##1}}}
\expandafter\def\csname PYG@tok@vg\endcsname{\def\PYG@tc##1{\textcolor[rgb]{0.73,0.38,0.84}{##1}}}
\expandafter\def\csname PYG@tok@m\endcsname{\def\PYG@tc##1{\textcolor[rgb]{0.13,0.50,0.31}{##1}}}
\expandafter\def\csname PYG@tok@mh\endcsname{\def\PYG@tc##1{\textcolor[rgb]{0.13,0.50,0.31}{##1}}}
\expandafter\def\csname PYG@tok@cs\endcsname{\def\PYG@tc##1{\textcolor[rgb]{0.25,0.50,0.56}{##1}}\def\PYG@bc##1{\setlength{\fboxsep}{0pt}\colorbox[rgb]{1.00,0.94,0.94}{\strut ##1}}}
\expandafter\def\csname PYG@tok@ge\endcsname{\let\PYG@it=\textit}
\expandafter\def\csname PYG@tok@vc\endcsname{\def\PYG@tc##1{\textcolor[rgb]{0.73,0.38,0.84}{##1}}}
\expandafter\def\csname PYG@tok@il\endcsname{\def\PYG@tc##1{\textcolor[rgb]{0.13,0.50,0.31}{##1}}}
\expandafter\def\csname PYG@tok@go\endcsname{\def\PYG@tc##1{\textcolor[rgb]{0.20,0.20,0.20}{##1}}}
\expandafter\def\csname PYG@tok@cp\endcsname{\def\PYG@tc##1{\textcolor[rgb]{0.00,0.44,0.13}{##1}}}
\expandafter\def\csname PYG@tok@gi\endcsname{\def\PYG@tc##1{\textcolor[rgb]{0.00,0.63,0.00}{##1}}}
\expandafter\def\csname PYG@tok@gh\endcsname{\let\PYG@bf=\textbf\def\PYG@tc##1{\textcolor[rgb]{0.00,0.00,0.50}{##1}}}
\expandafter\def\csname PYG@tok@ni\endcsname{\let\PYG@bf=\textbf\def\PYG@tc##1{\textcolor[rgb]{0.84,0.33,0.22}{##1}}}
\expandafter\def\csname PYG@tok@nl\endcsname{\let\PYG@bf=\textbf\def\PYG@tc##1{\textcolor[rgb]{0.00,0.13,0.44}{##1}}}
\expandafter\def\csname PYG@tok@nn\endcsname{\let\PYG@bf=\textbf\def\PYG@tc##1{\textcolor[rgb]{0.05,0.52,0.71}{##1}}}
\expandafter\def\csname PYG@tok@no\endcsname{\def\PYG@tc##1{\textcolor[rgb]{0.38,0.68,0.84}{##1}}}
\expandafter\def\csname PYG@tok@na\endcsname{\def\PYG@tc##1{\textcolor[rgb]{0.25,0.44,0.63}{##1}}}
\expandafter\def\csname PYG@tok@nb\endcsname{\def\PYG@tc##1{\textcolor[rgb]{0.00,0.44,0.13}{##1}}}
\expandafter\def\csname PYG@tok@nc\endcsname{\let\PYG@bf=\textbf\def\PYG@tc##1{\textcolor[rgb]{0.05,0.52,0.71}{##1}}}
\expandafter\def\csname PYG@tok@nd\endcsname{\let\PYG@bf=\textbf\def\PYG@tc##1{\textcolor[rgb]{0.33,0.33,0.33}{##1}}}
\expandafter\def\csname PYG@tok@ne\endcsname{\def\PYG@tc##1{\textcolor[rgb]{0.00,0.44,0.13}{##1}}}
\expandafter\def\csname PYG@tok@nf\endcsname{\def\PYG@tc##1{\textcolor[rgb]{0.02,0.16,0.49}{##1}}}
\expandafter\def\csname PYG@tok@si\endcsname{\let\PYG@it=\textit\def\PYG@tc##1{\textcolor[rgb]{0.44,0.63,0.82}{##1}}}
\expandafter\def\csname PYG@tok@s2\endcsname{\def\PYG@tc##1{\textcolor[rgb]{0.25,0.44,0.63}{##1}}}
\expandafter\def\csname PYG@tok@vi\endcsname{\def\PYG@tc##1{\textcolor[rgb]{0.73,0.38,0.84}{##1}}}
\expandafter\def\csname PYG@tok@nt\endcsname{\let\PYG@bf=\textbf\def\PYG@tc##1{\textcolor[rgb]{0.02,0.16,0.45}{##1}}}
\expandafter\def\csname PYG@tok@nv\endcsname{\def\PYG@tc##1{\textcolor[rgb]{0.73,0.38,0.84}{##1}}}
\expandafter\def\csname PYG@tok@s1\endcsname{\def\PYG@tc##1{\textcolor[rgb]{0.25,0.44,0.63}{##1}}}
\expandafter\def\csname PYG@tok@gp\endcsname{\let\PYG@bf=\textbf\def\PYG@tc##1{\textcolor[rgb]{0.78,0.36,0.04}{##1}}}
\expandafter\def\csname PYG@tok@sh\endcsname{\def\PYG@tc##1{\textcolor[rgb]{0.25,0.44,0.63}{##1}}}
\expandafter\def\csname PYG@tok@ow\endcsname{\let\PYG@bf=\textbf\def\PYG@tc##1{\textcolor[rgb]{0.00,0.44,0.13}{##1}}}
\expandafter\def\csname PYG@tok@sx\endcsname{\def\PYG@tc##1{\textcolor[rgb]{0.78,0.36,0.04}{##1}}}
\expandafter\def\csname PYG@tok@bp\endcsname{\def\PYG@tc##1{\textcolor[rgb]{0.00,0.44,0.13}{##1}}}
\expandafter\def\csname PYG@tok@c1\endcsname{\let\PYG@it=\textit\def\PYG@tc##1{\textcolor[rgb]{0.25,0.50,0.56}{##1}}}
\expandafter\def\csname PYG@tok@kc\endcsname{\let\PYG@bf=\textbf\def\PYG@tc##1{\textcolor[rgb]{0.00,0.44,0.13}{##1}}}
\expandafter\def\csname PYG@tok@c\endcsname{\let\PYG@it=\textit\def\PYG@tc##1{\textcolor[rgb]{0.25,0.50,0.56}{##1}}}
\expandafter\def\csname PYG@tok@mf\endcsname{\def\PYG@tc##1{\textcolor[rgb]{0.13,0.50,0.31}{##1}}}
\expandafter\def\csname PYG@tok@err\endcsname{\def\PYG@bc##1{\setlength{\fboxsep}{0pt}\fcolorbox[rgb]{1.00,0.00,0.00}{1,1,1}{\strut ##1}}}
\expandafter\def\csname PYG@tok@kd\endcsname{\let\PYG@bf=\textbf\def\PYG@tc##1{\textcolor[rgb]{0.00,0.44,0.13}{##1}}}
\expandafter\def\csname PYG@tok@ss\endcsname{\def\PYG@tc##1{\textcolor[rgb]{0.32,0.47,0.09}{##1}}}
\expandafter\def\csname PYG@tok@sr\endcsname{\def\PYG@tc##1{\textcolor[rgb]{0.14,0.33,0.53}{##1}}}
\expandafter\def\csname PYG@tok@mo\endcsname{\def\PYG@tc##1{\textcolor[rgb]{0.13,0.50,0.31}{##1}}}
\expandafter\def\csname PYG@tok@mi\endcsname{\def\PYG@tc##1{\textcolor[rgb]{0.13,0.50,0.31}{##1}}}
\expandafter\def\csname PYG@tok@kn\endcsname{\let\PYG@bf=\textbf\def\PYG@tc##1{\textcolor[rgb]{0.00,0.44,0.13}{##1}}}
\expandafter\def\csname PYG@tok@o\endcsname{\def\PYG@tc##1{\textcolor[rgb]{0.40,0.40,0.40}{##1}}}
\expandafter\def\csname PYG@tok@kr\endcsname{\let\PYG@bf=\textbf\def\PYG@tc##1{\textcolor[rgb]{0.00,0.44,0.13}{##1}}}
\expandafter\def\csname PYG@tok@s\endcsname{\def\PYG@tc##1{\textcolor[rgb]{0.25,0.44,0.63}{##1}}}
\expandafter\def\csname PYG@tok@kp\endcsname{\def\PYG@tc##1{\textcolor[rgb]{0.00,0.44,0.13}{##1}}}
\expandafter\def\csname PYG@tok@w\endcsname{\def\PYG@tc##1{\textcolor[rgb]{0.73,0.73,0.73}{##1}}}
\expandafter\def\csname PYG@tok@kt\endcsname{\def\PYG@tc##1{\textcolor[rgb]{0.56,0.13,0.00}{##1}}}
\expandafter\def\csname PYG@tok@sc\endcsname{\def\PYG@tc##1{\textcolor[rgb]{0.25,0.44,0.63}{##1}}}
\expandafter\def\csname PYG@tok@sb\endcsname{\def\PYG@tc##1{\textcolor[rgb]{0.25,0.44,0.63}{##1}}}
\expandafter\def\csname PYG@tok@k\endcsname{\let\PYG@bf=\textbf\def\PYG@tc##1{\textcolor[rgb]{0.00,0.44,0.13}{##1}}}
\expandafter\def\csname PYG@tok@se\endcsname{\let\PYG@bf=\textbf\def\PYG@tc##1{\textcolor[rgb]{0.25,0.44,0.63}{##1}}}
\expandafter\def\csname PYG@tok@sd\endcsname{\let\PYG@it=\textit\def\PYG@tc##1{\textcolor[rgb]{0.25,0.44,0.63}{##1}}}

\def\PYGZbs{\char`\\}
\def\PYGZus{\char`\_}
\def\PYGZob{\char`\{}
\def\PYGZcb{\char`\}}
\def\PYGZca{\char`\^}
\def\PYGZam{\char`\&}
\def\PYGZlt{\char`\<}
\def\PYGZgt{\char`\>}
\def\PYGZsh{\char`\#}
\def\PYGZpc{\char`\%}
\def\PYGZdl{\char`\$}
\def\PYGZhy{\char`\-}
\def\PYGZsq{\char`\'}
\def\PYGZdq{\char`\"}
\def\PYGZti{\char`\~}
% for compatibility with earlier versions
\def\PYGZat{@}
\def\PYGZlb{[}
\def\PYGZrb{]}
\makeatother

\begin{document}

\maketitle
\tableofcontents
\phantomsection\label{index::doc}


PyScaffold helps you to easily setup a new Python project, it is as easy as:

\begin{Verbatim}[commandchars=\\\{\}]
putup my\_project
\end{Verbatim}

This will create a new subdirectory \code{my\_project} and serve you a project
setup with git repository, setup.py, document and test folder ready for some
serious coding.

Type \code{putup -h} to learn about more configuration options. PyScaffold assumes
that you have \href{http://git-scm.com/}{Git} installed and set up on your PC,
meaning at least your name and email configured.
The scaffold of \code{my\_project} provides you with following
{\hyperref[features:features]{\emph{features}}}.


\chapter{Contents}
\label{index:pyscaffold}\label{index:contents}

\section{Features}
\label{features::doc}\label{features:features}\label{features:id1}

\subsection{Packaging}
\label{features:packaging}
Run \code{python setup.py sdist}, \code{python setup.py bdist} or
\code{python setup.py bdist\_wheel} to build a source, binary or wheel
distribution.


\subsection{Complete Git Integration}
\label{features:complete-git-integration}
Your project is already an initialised Git repository and \code{setup.py} uses
the information of tags to infer the version of your project with the help of
\href{https://github.com/warner/python-versioneer}{versioneer}.
To use this feature you need to tag with the format \code{vMAJOR.MINOR{[}.REVISION{]}}
, e.g. \code{v0.0.1} or \code{v0.1}. The prefix \code{v} is needed!
Run \code{python setup.py version} to retrieve the current \href{http://www.python.org/dev/peps/pep-0440/}{PEP440}-compliant version. This version
will be used when building a package and is also accessible through
\code{my\_project.\_\_version\_\_}.
The version will be \code{unknown} until you have added a first tag.


\subsection{Sphinx Documentation}
\label{features:sphinx-documentation}
Build the documentation with \code{python setup.py docs} and run doctests with
\code{python setup.py doctest}. Start editing the file \code{docs/index.rst} to
extend the documentation. The documentation also works with \href{https://readthedocs.org/}{Read the Docs}.


\subsection{Unittest \& Coverage}
\label{features:unittest-coverage}
Run \code{python setup.py test} to run all unittests defined in the subfolder
\code{tests} with the help of \href{http://pytest.org/}{py.test}. The py.test plugin
\href{https://github.com/schlamar/pytest-cov}{pytest-cov} is used to automatically
generate a coverage report. For usage with a continuous integration software
JUnit and Coverage XML output can be activated. Checkout \code{putup -h} for
details.


\subsection{Requirements Management}
\label{features:requirements-management}
Add the requirements of your project to the \code{requirements.txt} file which
will be automatically used by \code{setup.py}.


\subsection{Easy Updating}
\label{features:easy-updating}
Keep your project's scaffold up-to-date by applying
\code{putput -{-}update my\_project} when a new version of PyScaffold was released.
It may also be used to change the url, license and description setting.


\section{Contributing}
\label{contrib:contributing}\label{contrib::doc}
PyScaffold is developed by Blue Yonder developers to help automating and
standardizing the process of project setups.
You are very welcome to join in our effort if you would like to contribute.


\subsection{Bug Reports}
\label{contrib:bug-reports}
If you experience bugs or in general issues with PyScaffold, please file a bug
report to our \href{http://github.com/blue-yonder/pyscaffold/issues}{Bug Tracker}.


\subsection{Code}
\label{contrib:code}
If you would like to contribute to PyScaffold, fork the \href{https://github.com/blue-yonder/pyscaffold/}{main repository} on GitHub, then submit a
“pull request” (PR):
\begin{enumerate}
\item {} 
\href{https://github.com/signup/free}{Create an account} on GitHub if you do
not already have one.

\item {} 
Fork the project repository: click on the \emph{Fork} button near the top of the
page. This creates a copy of the code under your account on the GitHub server.

\item {} 
Clone this copy to your local disk:

\begin{Verbatim}[commandchars=\\\{\}]
git clone git@github.com:YourLogin/pyscaffold.git
\end{Verbatim}

\item {} 
Create a branch to hold your changes:

\begin{Verbatim}[commandchars=\\\{\}]
git checkout -b my-feature
\end{Verbatim}

and start making changes. Never work in the master branch!

\item {} 
Work on this copy, on your computer, using \href{http://git-scm.com/}{Git} to
do the version control. When you’re done editing, do:

\begin{Verbatim}[commandchars=\\\{\}]
git add modified\_files
git commit
\end{Verbatim}

to record your changes in Git, then push them to GitHub with:

\begin{Verbatim}[commandchars=\\\{\}]
git push -u origin my-feature
\end{Verbatim}

\end{enumerate}


\section{pyscaffold}
\label{_rst/modules:pyscaffold}\label{_rst/modules::doc}

\subsection{pyscaffold package}
\label{_rst/pyscaffold::doc}\label{_rst/pyscaffold:pyscaffold-package}

\subsubsection{Submodules}
\label{_rst/pyscaffold:submodules}

\subsubsection{pyscaffold.info module}
\label{_rst/pyscaffold:module-pyscaffold.info}\label{_rst/pyscaffold:pyscaffold-info-module}\index{pyscaffold.info (module)}\index{email() (in module pyscaffold.info)}

\begin{fulllineitems}
\phantomsection\label{_rst/pyscaffold:pyscaffold.info.email}\pysiglinewithargsret{\code{pyscaffold.info.}\bfcode{email}}{}{}
\end{fulllineitems}

\index{is\_git\_installed() (in module pyscaffold.info)}

\begin{fulllineitems}
\phantomsection\label{_rst/pyscaffold:pyscaffold.info.is_git_installed}\pysiglinewithargsret{\code{pyscaffold.info.}\bfcode{is\_git\_installed}}{}{}
\end{fulllineitems}

\index{project() (in module pyscaffold.info)}

\begin{fulllineitems}
\phantomsection\label{_rst/pyscaffold:pyscaffold.info.project}\pysiglinewithargsret{\code{pyscaffold.info.}\bfcode{project}}{\emph{args}}{}
\end{fulllineitems}

\index{username() (in module pyscaffold.info)}

\begin{fulllineitems}
\phantomsection\label{_rst/pyscaffold:pyscaffold.info.username}\pysiglinewithargsret{\code{pyscaffold.info.}\bfcode{username}}{}{}
\end{fulllineitems}



\subsubsection{pyscaffold.repo module}
\label{_rst/pyscaffold:module-pyscaffold.repo}\label{_rst/pyscaffold:pyscaffold-repo-module}\index{pyscaffold.repo (module)}\index{git\_tree\_add() (in module pyscaffold.repo)}

\begin{fulllineitems}
\phantomsection\label{_rst/pyscaffold:pyscaffold.repo.git_tree_add}\pysiglinewithargsret{\code{pyscaffold.repo.}\bfcode{git\_tree\_add}}{\emph{struct}, \emph{prefix='`}}{}
\end{fulllineitems}

\index{init\_commit\_repo() (in module pyscaffold.repo)}

\begin{fulllineitems}
\phantomsection\label{_rst/pyscaffold:pyscaffold.repo.init_commit_repo}\pysiglinewithargsret{\code{pyscaffold.repo.}\bfcode{init\_commit\_repo}}{\emph{project}, \emph{struct}}{}
\end{fulllineitems}

\index{is\_git\_repo() (in module pyscaffold.repo)}

\begin{fulllineitems}
\phantomsection\label{_rst/pyscaffold:pyscaffold.repo.is_git_repo}\pysiglinewithargsret{\code{pyscaffold.repo.}\bfcode{is\_git\_repo}}{\emph{folder}}{}
\end{fulllineitems}



\subsubsection{pyscaffold.runner module}
\label{_rst/pyscaffold:module-pyscaffold.runner}\label{_rst/pyscaffold:pyscaffold-runner-module}\index{pyscaffold.runner (module)}\index{main() (in module pyscaffold.runner)}

\begin{fulllineitems}
\phantomsection\label{_rst/pyscaffold:pyscaffold.runner.main}\pysiglinewithargsret{\code{pyscaffold.runner.}\bfcode{main}}{\emph{args}}{}
\end{fulllineitems}

\index{parse\_args() (in module pyscaffold.runner)}

\begin{fulllineitems}
\phantomsection\label{_rst/pyscaffold:pyscaffold.runner.parse_args}\pysiglinewithargsret{\code{pyscaffold.runner.}\bfcode{parse\_args}}{\emph{args}}{}
\end{fulllineitems}

\index{run() (in module pyscaffold.runner)}

\begin{fulllineitems}
\phantomsection\label{_rst/pyscaffold:pyscaffold.runner.run}\pysiglinewithargsret{\code{pyscaffold.runner.}\bfcode{run}}{\emph{*args}, \emph{**kwargs}}{}
Entry point for setup.py

\end{fulllineitems}



\subsubsection{pyscaffold.structure module}
\label{_rst/pyscaffold:module-pyscaffold.structure}\label{_rst/pyscaffold:pyscaffold-structure-module}\index{pyscaffold.structure (module)}\index{create\_structure() (in module pyscaffold.structure)}

\begin{fulllineitems}
\phantomsection\label{_rst/pyscaffold:pyscaffold.structure.create_structure}\pysiglinewithargsret{\code{pyscaffold.structure.}\bfcode{create\_structure}}{\emph{struct}, \emph{prefix=None}, \emph{update=False}}{}
\end{fulllineitems}

\index{make\_structure() (in module pyscaffold.structure)}

\begin{fulllineitems}
\phantomsection\label{_rst/pyscaffold:pyscaffold.structure.make_structure}\pysiglinewithargsret{\code{pyscaffold.structure.}\bfcode{make\_structure}}{\emph{args}}{}
\end{fulllineitems}

\index{set\_default\_args() (in module pyscaffold.structure)}

\begin{fulllineitems}
\phantomsection\label{_rst/pyscaffold:pyscaffold.structure.set_default_args}\pysiglinewithargsret{\code{pyscaffold.structure.}\bfcode{set\_default\_args}}{\emph{args}}{}
\end{fulllineitems}



\subsubsection{pyscaffold.templates module}
\label{_rst/pyscaffold:pyscaffold-templates-module}\label{_rst/pyscaffold:module-pyscaffold.templates}\index{pyscaffold.templates (module)}\index{authors() (in module pyscaffold.templates)}

\begin{fulllineitems}
\phantomsection\label{_rst/pyscaffold:pyscaffold.templates.authors}\pysiglinewithargsret{\code{pyscaffold.templates.}\bfcode{authors}}{\emph{args}}{}
\end{fulllineitems}

\index{copying() (in module pyscaffold.templates)}

\begin{fulllineitems}
\phantomsection\label{_rst/pyscaffold:pyscaffold.templates.copying}\pysiglinewithargsret{\code{pyscaffold.templates.}\bfcode{copying}}{\emph{args}}{}
\end{fulllineitems}

\index{coveragerc() (in module pyscaffold.templates)}

\begin{fulllineitems}
\phantomsection\label{_rst/pyscaffold:pyscaffold.templates.coveragerc}\pysiglinewithargsret{\code{pyscaffold.templates.}\bfcode{coveragerc}}{\emph{args}}{}
\end{fulllineitems}

\index{get\_template() (in module pyscaffold.templates)}

\begin{fulllineitems}
\phantomsection\label{_rst/pyscaffold:pyscaffold.templates.get_template}\pysiglinewithargsret{\code{pyscaffold.templates.}\bfcode{get\_template}}{\emph{name}}{}
\end{fulllineitems}

\index{gitignore() (in module pyscaffold.templates)}

\begin{fulllineitems}
\phantomsection\label{_rst/pyscaffold:pyscaffold.templates.gitignore}\pysiglinewithargsret{\code{pyscaffold.templates.}\bfcode{gitignore}}{\emph{args}}{}
\end{fulllineitems}

\index{gitignore\_empty() (in module pyscaffold.templates)}

\begin{fulllineitems}
\phantomsection\label{_rst/pyscaffold:pyscaffold.templates.gitignore_empty}\pysiglinewithargsret{\code{pyscaffold.templates.}\bfcode{gitignore\_empty}}{\emph{args}}{}
\end{fulllineitems}

\index{init() (in module pyscaffold.templates)}

\begin{fulllineitems}
\phantomsection\label{_rst/pyscaffold:pyscaffold.templates.init}\pysiglinewithargsret{\code{pyscaffold.templates.}\bfcode{init}}{\emph{args}}{}
\end{fulllineitems}

\index{manifest\_in() (in module pyscaffold.templates)}

\begin{fulllineitems}
\phantomsection\label{_rst/pyscaffold:pyscaffold.templates.manifest_in}\pysiglinewithargsret{\code{pyscaffold.templates.}\bfcode{manifest\_in}}{\emph{args}}{}
\end{fulllineitems}

\index{readme() (in module pyscaffold.templates)}

\begin{fulllineitems}
\phantomsection\label{_rst/pyscaffold:pyscaffold.templates.readme}\pysiglinewithargsret{\code{pyscaffold.templates.}\bfcode{readme}}{\emph{args}}{}
\end{fulllineitems}

\index{requirements() (in module pyscaffold.templates)}

\begin{fulllineitems}
\phantomsection\label{_rst/pyscaffold:pyscaffold.templates.requirements}\pysiglinewithargsret{\code{pyscaffold.templates.}\bfcode{requirements}}{\emph{args}}{}
\end{fulllineitems}

\index{setup() (in module pyscaffold.templates)}

\begin{fulllineitems}
\phantomsection\label{_rst/pyscaffold:pyscaffold.templates.setup}\pysiglinewithargsret{\code{pyscaffold.templates.}\bfcode{setup}}{\emph{args}}{}
\end{fulllineitems}

\index{sphinx\_conf() (in module pyscaffold.templates)}

\begin{fulllineitems}
\phantomsection\label{_rst/pyscaffold:pyscaffold.templates.sphinx_conf}\pysiglinewithargsret{\code{pyscaffold.templates.}\bfcode{sphinx\_conf}}{\emph{args}}{}
\end{fulllineitems}

\index{sphinx\_index() (in module pyscaffold.templates)}

\begin{fulllineitems}
\phantomsection\label{_rst/pyscaffold:pyscaffold.templates.sphinx_index}\pysiglinewithargsret{\code{pyscaffold.templates.}\bfcode{sphinx\_index}}{\emph{args}}{}
\end{fulllineitems}

\index{sphinx\_makefile() (in module pyscaffold.templates)}

\begin{fulllineitems}
\phantomsection\label{_rst/pyscaffold:pyscaffold.templates.sphinx_makefile}\pysiglinewithargsret{\code{pyscaffold.templates.}\bfcode{sphinx\_makefile}}{\emph{args}}{}
\end{fulllineitems}

\index{version() (in module pyscaffold.templates)}

\begin{fulllineitems}
\phantomsection\label{_rst/pyscaffold:pyscaffold.templates.version}\pysiglinewithargsret{\code{pyscaffold.templates.}\bfcode{version}}{\emph{args}}{}
\end{fulllineitems}

\index{versioneer() (in module pyscaffold.templates)}

\begin{fulllineitems}
\phantomsection\label{_rst/pyscaffold:pyscaffold.templates.versioneer}\pysiglinewithargsret{\code{pyscaffold.templates.}\bfcode{versioneer}}{\emph{args}}{}
\end{fulllineitems}



\subsubsection{pyscaffold.utils module}
\label{_rst/pyscaffold:pyscaffold-utils-module}\label{_rst/pyscaffold:module-pyscaffold.utils}\index{pyscaffold.utils (module)}\index{ObjKeeper (class in pyscaffold.utils)}

\begin{fulllineitems}
\phantomsection\label{_rst/pyscaffold:pyscaffold.utils.ObjKeeper}\pysiglinewithargsret{\strong{class }\code{pyscaffold.utils.}\bfcode{ObjKeeper}}{\emph{name}, \emph{bases}, \emph{dct}}{}
Bases: \code{type}
\index{instances (pyscaffold.utils.ObjKeeper attribute)}

\begin{fulllineitems}
\phantomsection\label{_rst/pyscaffold:pyscaffold.utils.ObjKeeper.instances}\pysigline{\bfcode{instances}\strong{ = \{\}}}
\end{fulllineitems}


\end{fulllineitems}

\index{capture\_objs() (in module pyscaffold.utils)}

\begin{fulllineitems}
\phantomsection\label{_rst/pyscaffold:pyscaffold.utils.capture_objs}\pysiglinewithargsret{\code{pyscaffold.utils.}\bfcode{capture\_objs}}{\emph{cls}}{}
\end{fulllineitems}

\index{chdir() (in module pyscaffold.utils)}

\begin{fulllineitems}
\phantomsection\label{_rst/pyscaffold:pyscaffold.utils.chdir}\pysiglinewithargsret{\code{pyscaffold.utils.}\bfcode{chdir}}{\emph{*args}, \emph{**kwds}}{}
\end{fulllineitems}

\index{exceptions2exit() (in module pyscaffold.utils)}

\begin{fulllineitems}
\phantomsection\label{_rst/pyscaffold:pyscaffold.utils.exceptions2exit}\pysiglinewithargsret{\code{pyscaffold.utils.}\bfcode{exceptions2exit}}{\emph{exception\_list}}{}
\end{fulllineitems}

\index{is\_valid\_identifier() (in module pyscaffold.utils)}

\begin{fulllineitems}
\phantomsection\label{_rst/pyscaffold:pyscaffold.utils.is_valid_identifier}\pysiglinewithargsret{\code{pyscaffold.utils.}\bfcode{is\_valid\_identifier}}{\emph{string}}{}
\end{fulllineitems}

\index{list2str() (in module pyscaffold.utils)}

\begin{fulllineitems}
\phantomsection\label{_rst/pyscaffold:pyscaffold.utils.list2str}\pysiglinewithargsret{\code{pyscaffold.utils.}\bfcode{list2str}}{\emph{lst}, \emph{indent=0}}{}
\end{fulllineitems}

\index{make\_valid\_identifier() (in module pyscaffold.utils)}

\begin{fulllineitems}
\phantomsection\label{_rst/pyscaffold:pyscaffold.utils.make_valid_identifier}\pysiglinewithargsret{\code{pyscaffold.utils.}\bfcode{make\_valid\_identifier}}{\emph{string}}{}
\end{fulllineitems}

\index{safe\_get() (in module pyscaffold.utils)}

\begin{fulllineitems}
\phantomsection\label{_rst/pyscaffold:pyscaffold.utils.safe_get}\pysiglinewithargsret{\code{pyscaffold.utils.}\bfcode{safe\_get}}{\emph{namespace}, \emph{attr}}{}
\end{fulllineitems}

\index{safe\_set() (in module pyscaffold.utils)}

\begin{fulllineitems}
\phantomsection\label{_rst/pyscaffold:pyscaffold.utils.safe_set}\pysiglinewithargsret{\code{pyscaffold.utils.}\bfcode{safe\_set}}{\emph{namespace}, \emph{attr}, \emph{value}}{}
\end{fulllineitems}



\subsubsection{Module contents}
\label{_rst/pyscaffold:module-contents}\label{_rst/pyscaffold:module-pyscaffold}\index{pyscaffold (module)}

\chapter{Indices and tables}
\label{index:indices-and-tables}\begin{itemize}
\item {} 
\emph{genindex}

\item {} 
\emph{modindex}

\item {} 
\emph{search}

\end{itemize}


\renewcommand{\indexname}{Python Module Index}
\begin{theindex}
\def\bigletter#1{{\Large\sffamily#1}\nopagebreak\vspace{1mm}}
\bigletter{p}
\item {\texttt{pyscaffold}}, \pageref{_rst/pyscaffold:module-pyscaffold}
\item {\texttt{pyscaffold.info}}, \pageref{_rst/pyscaffold:module-pyscaffold.info}
\item {\texttt{pyscaffold.repo}}, \pageref{_rst/pyscaffold:module-pyscaffold.repo}
\item {\texttt{pyscaffold.runner}}, \pageref{_rst/pyscaffold:module-pyscaffold.runner}
\item {\texttt{pyscaffold.structure}}, \pageref{_rst/pyscaffold:module-pyscaffold.structure}
\item {\texttt{pyscaffold.templates}}, \pageref{_rst/pyscaffold:module-pyscaffold.templates}
\item {\texttt{pyscaffold.utils}}, \pageref{_rst/pyscaffold:module-pyscaffold.utils}
\end{theindex}

\renewcommand{\indexname}{Index}
\printindex
\end{document}
