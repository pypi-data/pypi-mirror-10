% Generated by Sphinx.
\def\sphinxdocclass{report}
\documentclass[letterpaper,10pt,english]{sphinxmanual}
\usepackage[utf8]{inputenc}
\DeclareUnicodeCharacter{00A0}{\nobreakspace}
\usepackage{cmap}
\usepackage[T1]{fontenc}
\usepackage{babel}
\usepackage{times}
\usepackage[Bjarne]{fncychap}
\usepackage{longtable}
\usepackage{sphinx}
\usepackage{multirow}


\title{pyscaffold Documentation}
\date{September 12, 2014}
\release{1.0.post.dev9}
\author{Blue Yonder}
\newcommand{\sphinxlogo}{}
\renewcommand{\releasename}{Release}
\makeindex

\makeatletter
\def\PYG@reset{\let\PYG@it=\relax \let\PYG@bf=\relax%
    \let\PYG@ul=\relax \let\PYG@tc=\relax%
    \let\PYG@bc=\relax \let\PYG@ff=\relax}
\def\PYG@tok#1{\csname PYG@tok@#1\endcsname}
\def\PYG@toks#1+{\ifx\relax#1\empty\else%
    \PYG@tok{#1}\expandafter\PYG@toks\fi}
\def\PYG@do#1{\PYG@bc{\PYG@tc{\PYG@ul{%
    \PYG@it{\PYG@bf{\PYG@ff{#1}}}}}}}
\def\PYG#1#2{\PYG@reset\PYG@toks#1+\relax+\PYG@do{#2}}

\expandafter\def\csname PYG@tok@gd\endcsname{\def\PYG@tc##1{\textcolor[rgb]{0.63,0.00,0.00}{##1}}}
\expandafter\def\csname PYG@tok@gu\endcsname{\let\PYG@bf=\textbf\def\PYG@tc##1{\textcolor[rgb]{0.50,0.00,0.50}{##1}}}
\expandafter\def\csname PYG@tok@gt\endcsname{\def\PYG@tc##1{\textcolor[rgb]{0.00,0.27,0.87}{##1}}}
\expandafter\def\csname PYG@tok@gs\endcsname{\let\PYG@bf=\textbf}
\expandafter\def\csname PYG@tok@gr\endcsname{\def\PYG@tc##1{\textcolor[rgb]{1.00,0.00,0.00}{##1}}}
\expandafter\def\csname PYG@tok@cm\endcsname{\let\PYG@it=\textit\def\PYG@tc##1{\textcolor[rgb]{0.25,0.50,0.56}{##1}}}
\expandafter\def\csname PYG@tok@vg\endcsname{\def\PYG@tc##1{\textcolor[rgb]{0.73,0.38,0.84}{##1}}}
\expandafter\def\csname PYG@tok@m\endcsname{\def\PYG@tc##1{\textcolor[rgb]{0.13,0.50,0.31}{##1}}}
\expandafter\def\csname PYG@tok@mh\endcsname{\def\PYG@tc##1{\textcolor[rgb]{0.13,0.50,0.31}{##1}}}
\expandafter\def\csname PYG@tok@cs\endcsname{\def\PYG@tc##1{\textcolor[rgb]{0.25,0.50,0.56}{##1}}\def\PYG@bc##1{\setlength{\fboxsep}{0pt}\colorbox[rgb]{1.00,0.94,0.94}{\strut ##1}}}
\expandafter\def\csname PYG@tok@ge\endcsname{\let\PYG@it=\textit}
\expandafter\def\csname PYG@tok@vc\endcsname{\def\PYG@tc##1{\textcolor[rgb]{0.73,0.38,0.84}{##1}}}
\expandafter\def\csname PYG@tok@il\endcsname{\def\PYG@tc##1{\textcolor[rgb]{0.13,0.50,0.31}{##1}}}
\expandafter\def\csname PYG@tok@go\endcsname{\def\PYG@tc##1{\textcolor[rgb]{0.20,0.20,0.20}{##1}}}
\expandafter\def\csname PYG@tok@cp\endcsname{\def\PYG@tc##1{\textcolor[rgb]{0.00,0.44,0.13}{##1}}}
\expandafter\def\csname PYG@tok@gi\endcsname{\def\PYG@tc##1{\textcolor[rgb]{0.00,0.63,0.00}{##1}}}
\expandafter\def\csname PYG@tok@gh\endcsname{\let\PYG@bf=\textbf\def\PYG@tc##1{\textcolor[rgb]{0.00,0.00,0.50}{##1}}}
\expandafter\def\csname PYG@tok@ni\endcsname{\let\PYG@bf=\textbf\def\PYG@tc##1{\textcolor[rgb]{0.84,0.33,0.22}{##1}}}
\expandafter\def\csname PYG@tok@nl\endcsname{\let\PYG@bf=\textbf\def\PYG@tc##1{\textcolor[rgb]{0.00,0.13,0.44}{##1}}}
\expandafter\def\csname PYG@tok@nn\endcsname{\let\PYG@bf=\textbf\def\PYG@tc##1{\textcolor[rgb]{0.05,0.52,0.71}{##1}}}
\expandafter\def\csname PYG@tok@no\endcsname{\def\PYG@tc##1{\textcolor[rgb]{0.38,0.68,0.84}{##1}}}
\expandafter\def\csname PYG@tok@na\endcsname{\def\PYG@tc##1{\textcolor[rgb]{0.25,0.44,0.63}{##1}}}
\expandafter\def\csname PYG@tok@nb\endcsname{\def\PYG@tc##1{\textcolor[rgb]{0.00,0.44,0.13}{##1}}}
\expandafter\def\csname PYG@tok@nc\endcsname{\let\PYG@bf=\textbf\def\PYG@tc##1{\textcolor[rgb]{0.05,0.52,0.71}{##1}}}
\expandafter\def\csname PYG@tok@nd\endcsname{\let\PYG@bf=\textbf\def\PYG@tc##1{\textcolor[rgb]{0.33,0.33,0.33}{##1}}}
\expandafter\def\csname PYG@tok@ne\endcsname{\def\PYG@tc##1{\textcolor[rgb]{0.00,0.44,0.13}{##1}}}
\expandafter\def\csname PYG@tok@nf\endcsname{\def\PYG@tc##1{\textcolor[rgb]{0.02,0.16,0.49}{##1}}}
\expandafter\def\csname PYG@tok@si\endcsname{\let\PYG@it=\textit\def\PYG@tc##1{\textcolor[rgb]{0.44,0.63,0.82}{##1}}}
\expandafter\def\csname PYG@tok@s2\endcsname{\def\PYG@tc##1{\textcolor[rgb]{0.25,0.44,0.63}{##1}}}
\expandafter\def\csname PYG@tok@vi\endcsname{\def\PYG@tc##1{\textcolor[rgb]{0.73,0.38,0.84}{##1}}}
\expandafter\def\csname PYG@tok@nt\endcsname{\let\PYG@bf=\textbf\def\PYG@tc##1{\textcolor[rgb]{0.02,0.16,0.45}{##1}}}
\expandafter\def\csname PYG@tok@nv\endcsname{\def\PYG@tc##1{\textcolor[rgb]{0.73,0.38,0.84}{##1}}}
\expandafter\def\csname PYG@tok@s1\endcsname{\def\PYG@tc##1{\textcolor[rgb]{0.25,0.44,0.63}{##1}}}
\expandafter\def\csname PYG@tok@gp\endcsname{\let\PYG@bf=\textbf\def\PYG@tc##1{\textcolor[rgb]{0.78,0.36,0.04}{##1}}}
\expandafter\def\csname PYG@tok@sh\endcsname{\def\PYG@tc##1{\textcolor[rgb]{0.25,0.44,0.63}{##1}}}
\expandafter\def\csname PYG@tok@ow\endcsname{\let\PYG@bf=\textbf\def\PYG@tc##1{\textcolor[rgb]{0.00,0.44,0.13}{##1}}}
\expandafter\def\csname PYG@tok@sx\endcsname{\def\PYG@tc##1{\textcolor[rgb]{0.78,0.36,0.04}{##1}}}
\expandafter\def\csname PYG@tok@bp\endcsname{\def\PYG@tc##1{\textcolor[rgb]{0.00,0.44,0.13}{##1}}}
\expandafter\def\csname PYG@tok@c1\endcsname{\let\PYG@it=\textit\def\PYG@tc##1{\textcolor[rgb]{0.25,0.50,0.56}{##1}}}
\expandafter\def\csname PYG@tok@kc\endcsname{\let\PYG@bf=\textbf\def\PYG@tc##1{\textcolor[rgb]{0.00,0.44,0.13}{##1}}}
\expandafter\def\csname PYG@tok@c\endcsname{\let\PYG@it=\textit\def\PYG@tc##1{\textcolor[rgb]{0.25,0.50,0.56}{##1}}}
\expandafter\def\csname PYG@tok@mf\endcsname{\def\PYG@tc##1{\textcolor[rgb]{0.13,0.50,0.31}{##1}}}
\expandafter\def\csname PYG@tok@err\endcsname{\def\PYG@bc##1{\setlength{\fboxsep}{0pt}\fcolorbox[rgb]{1.00,0.00,0.00}{1,1,1}{\strut ##1}}}
\expandafter\def\csname PYG@tok@kd\endcsname{\let\PYG@bf=\textbf\def\PYG@tc##1{\textcolor[rgb]{0.00,0.44,0.13}{##1}}}
\expandafter\def\csname PYG@tok@ss\endcsname{\def\PYG@tc##1{\textcolor[rgb]{0.32,0.47,0.09}{##1}}}
\expandafter\def\csname PYG@tok@sr\endcsname{\def\PYG@tc##1{\textcolor[rgb]{0.14,0.33,0.53}{##1}}}
\expandafter\def\csname PYG@tok@mo\endcsname{\def\PYG@tc##1{\textcolor[rgb]{0.13,0.50,0.31}{##1}}}
\expandafter\def\csname PYG@tok@mi\endcsname{\def\PYG@tc##1{\textcolor[rgb]{0.13,0.50,0.31}{##1}}}
\expandafter\def\csname PYG@tok@kn\endcsname{\let\PYG@bf=\textbf\def\PYG@tc##1{\textcolor[rgb]{0.00,0.44,0.13}{##1}}}
\expandafter\def\csname PYG@tok@o\endcsname{\def\PYG@tc##1{\textcolor[rgb]{0.40,0.40,0.40}{##1}}}
\expandafter\def\csname PYG@tok@kr\endcsname{\let\PYG@bf=\textbf\def\PYG@tc##1{\textcolor[rgb]{0.00,0.44,0.13}{##1}}}
\expandafter\def\csname PYG@tok@s\endcsname{\def\PYG@tc##1{\textcolor[rgb]{0.25,0.44,0.63}{##1}}}
\expandafter\def\csname PYG@tok@kp\endcsname{\def\PYG@tc##1{\textcolor[rgb]{0.00,0.44,0.13}{##1}}}
\expandafter\def\csname PYG@tok@w\endcsname{\def\PYG@tc##1{\textcolor[rgb]{0.73,0.73,0.73}{##1}}}
\expandafter\def\csname PYG@tok@kt\endcsname{\def\PYG@tc##1{\textcolor[rgb]{0.56,0.13,0.00}{##1}}}
\expandafter\def\csname PYG@tok@sc\endcsname{\def\PYG@tc##1{\textcolor[rgb]{0.25,0.44,0.63}{##1}}}
\expandafter\def\csname PYG@tok@sb\endcsname{\def\PYG@tc##1{\textcolor[rgb]{0.25,0.44,0.63}{##1}}}
\expandafter\def\csname PYG@tok@k\endcsname{\let\PYG@bf=\textbf\def\PYG@tc##1{\textcolor[rgb]{0.00,0.44,0.13}{##1}}}
\expandafter\def\csname PYG@tok@se\endcsname{\let\PYG@bf=\textbf\def\PYG@tc##1{\textcolor[rgb]{0.25,0.44,0.63}{##1}}}
\expandafter\def\csname PYG@tok@sd\endcsname{\let\PYG@it=\textit\def\PYG@tc##1{\textcolor[rgb]{0.25,0.44,0.63}{##1}}}

\def\PYGZbs{\char`\\}
\def\PYGZus{\char`\_}
\def\PYGZob{\char`\{}
\def\PYGZcb{\char`\}}
\def\PYGZca{\char`\^}
\def\PYGZam{\char`\&}
\def\PYGZlt{\char`\<}
\def\PYGZgt{\char`\>}
\def\PYGZsh{\char`\#}
\def\PYGZpc{\char`\%}
\def\PYGZdl{\char`\$}
\def\PYGZhy{\char`\-}
\def\PYGZsq{\char`\'}
\def\PYGZdq{\char`\"}
\def\PYGZti{\char`\~}
% for compatibility with earlier versions
\def\PYGZat{@}
\def\PYGZlb{[}
\def\PYGZrb{]}
\makeatother

\begin{document}

\maketitle
\tableofcontents
\phantomsection\label{index::doc}


PyScaffold helps you to easily setup a new Python project, it is as easy as:

\begin{Verbatim}[commandchars=\\\{\}]
putup my\_project
\end{Verbatim}

This will create a new subdirectory \code{my\_project} and serve you a project
setup with git repository, setup.py, document and test folder ready for some
serious coding.

Type \code{putup -h} to learn about more configuration options. PyScaffold assumes
that you have \href{http://git-scm.com/}{Git} installed and set up on your PC,
meaning at least your name and email configured.
The scaffold of \code{my\_project} provides you with a lot of
{\hyperref[features:features]{\emph{features}}}. PyScaffold is compatible with Python 2.7, 3.3 and
3.4.


\chapter{Contents}
\label{index:pyscaffold}\label{index:contents}

\section{Features}
\label{features::doc}\label{features:features}\label{features:id1}

\subsection{Packaging}
\label{features:packaging}
Run \code{python setup.py sdist}, \code{python setup.py bdist} or
\code{python setup.py bdist\_wheel} to build a source, binary or wheel
distribution.


\subsection{Complete Git Integration}
\label{features:complete-git-integration}
Your project is already an initialised Git repository and \code{setup.py} uses
the information of tags to infer the version of your project with the help of
\href{https://github.com/warner/python-versioneer}{versioneer}.
To use this feature you need to tag with the format \code{vMAJOR.MINOR{[}.REVISION{]}}
, e.g. \code{v0.0.1} or \code{v0.1}. The prefix \code{v} is needed!
Run \code{python setup.py version} to retrieve the current \href{http://www.python.org/dev/peps/pep-0440/}{PEP440}-compliant version. This version
will be used when building a package and is also accessible through
\code{my\_project.\_\_version\_\_}.
The version will be \code{unknown} until you have added a first tag.


\subsection{Sphinx Documentation}
\label{features:sphinx-documentation}
Build the documentation with \code{python setup.py docs} and run doctests with
\code{python setup.py doctest}. Start editing the file \code{docs/index.rst} to
extend the documentation. The documentation also works with \href{https://readthedocs.org/}{Read the Docs}.


\subsection{Unittest \& Coverage}
\label{features:unittest-coverage}
Run \code{python setup.py test} to run all unittests defined in the subfolder
\code{tests} with the help of \href{http://pytest.org/}{py.test}. The py.test plugin
\href{https://github.com/schlamar/pytest-cov}{pytest-cov} is used to automatically
generate a coverage report. For usage with a continuous integration software
JUnit and Coverage XML output can be activated. Checkout \code{putup -h} for
details. Use the flag \code{-{-}with-travis} to generate templates of the
\href{https://travis-ci.org/}{Travis} configuration files \code{.travis.yml} and
\code{tests/travis\_install.sh} which even features the coverage and stats system
\href{https://coveralls.io/}{Coveralls}.


\subsection{Requirements Management}
\label{features:requirements-management}
Add the requirements of your project to the \code{requirements.txt} file which
will be automatically used by \code{setup.py}.


\subsection{Django}
\label{features:django}
Create a \href{https://www.djangoproject.com/}{Django project} with the flag
\code{-{-}with-django} which is equivalent to
\code{django-admin.py startproject my\_project} enhanced by PyScaffold's features.


\subsection{Easy Updating}
\label{features:easy-updating}
Keep your project's scaffold up-to-date by applying
\code{putput -{-}update my\_project} when a new version of PyScaffold was released.
It may also be used to change the url, license and description setting.
An update will only overwrite files that are not often altered by users like
setup.py, versioneer.py etc. To update all files use \code{-{-}update -{-}force}.
An existing project that was not setup with PyScaffold can be converted with
\code{putup -{-}force existing\_project}. The force option is completely save to use
since the git repository of the existing project is not touched!


\section{Installation}
\label{install:installation}\label{install::doc}

\subsection{Requirements}
\label{install:requirements}
The installation of PyScaffold requires:
\begin{itemize}
\item {} 
\href{https://pypi.python.org/pypi/setuptools/}{setuptools}

\item {} 
\href{https://pypi.python.org/pypi/six}{six}

\item {} 
\href{https://pypi.python.org/pypi/pytest/}{pytest}

\item {} 
\href{https://pypi.python.org/pypi/pytest-cov/}{pytest-cov}

\end{itemize}

Additionally, if you want to create a Django project:
\begin{itemize}
\item {} 
\href{https://pypi.python.org/pypi/Django/}{Django}

\end{itemize}

\begin{notice}{note}{Note:}
In most cases only Django needs to be installed manually since PyScaffold
will download and install its requirements automatically when using
\code{pip}. Once exception might be \emph{setuptools}. If you use a current version
of \href{http://docs.python-guide.org/en/latest/dev/virtualenvs/}{Virtual Environments} as development environment a current version should already
be installed. In case you are using the system installation of Python
from your Linux distribution make sure \emph{setuptools} is installed.
To install it on Debian or Ubuntu:

\begin{Verbatim}[commandchars=\\\{\}]
sudo apt-get install python-setuptools
\end{Verbatim}
\end{notice}


\subsection{Installation}
\label{install:id1}
If you have \code{pip} installed, then simply type:

\begin{Verbatim}[commandchars=\\\{\}]
pip install --upgrade pyscaffold
\end{Verbatim}

to get the lastest stable version. The most recent development version can be
installed with:

\begin{Verbatim}[commandchars=\\\{\}]
pip install --pre --upgrade pyscaffold
\end{Verbatim}

Using \code{pip} also has the advantage that all requirements are automatically
installed.


\section{Contributing}
\label{contrib:contributing}\label{contrib::doc}
PyScaffold is developed by \href{http://www.blue-yonder.com/en/}{Blue Yonder}
developers to help automating and standardizing the process of project setups.
You are very welcome to join in our effort if you would like to contribute.


\subsection{Bug Reports}
\label{contrib:bug-reports}
If you experience bugs or in general issues with PyScaffold, please file a bug
report to our \href{http://github.com/blue-yonder/pyscaffold/issues}{Bug Tracker}.


\subsection{Code}
\label{contrib:code}
If you would like to contribute to PyScaffold, fork the \href{https://github.com/blue-yonder/pyscaffold/}{main repository} on GitHub, then submit a
“pull request” (PR):
\begin{enumerate}
\item {} 
\href{https://github.com/signup/free}{Create an account} on GitHub if you do
not already have one.

\item {} 
Fork the project repository: click on the \emph{Fork} button near the top of the
page. This creates a copy of the code under your account on the GitHub server.

\item {} 
Clone this copy to your local disk:

\begin{Verbatim}[commandchars=\\\{\}]
git clone git@github.com:YourLogin/pyscaffold.git
\end{Verbatim}

\item {} 
Create a branch to hold your changes:

\begin{Verbatim}[commandchars=\\\{\}]
git checkout -b my-feature
\end{Verbatim}

and start making changes. Never work in the master branch!

\item {} 
Work on this copy, on your computer, using \href{http://git-scm.com/}{Git} to
do the version control. When you’re done editing, do:

\begin{Verbatim}[commandchars=\\\{\}]
git add modified\_files
git commit
\end{Verbatim}

to record your changes in Git, then push them to GitHub with:

\begin{Verbatim}[commandchars=\\\{\}]
git push -u origin my-feature
\end{Verbatim}

\end{enumerate}


\section{License}
\label{license::doc}\label{license:license}
\begin{Verbatim}[commandchars=\\\{\}]
Copyright (c) 2014, Blue Yonder GmbH.
All rights reserved.


Redistribution and use in source and binary forms, with or without
modification, are permitted provided that the following conditions are met:

  a. Redistributions of source code must retain the above copyright notice,
     this list of conditions and the following disclaimer.
  b. Redistributions in binary form must reproduce the above copyright
     notice, this list of conditions and the following disclaimer in the
     documentation and/or other materials provided with the distribution.
  c. Neither the name of the PyScaffold developers nor the names of
     its contributors may be used to endorse or promote products
     derived from this software without specific prior written
     permission.


THIS SOFTWARE IS PROVIDED BY THE COPYRIGHT HOLDERS AND CONTRIBUTORS "AS IS"
AND ANY EXPRESS OR IMPLIED WARRANTIES, INCLUDING, BUT NOT LIMITED TO, THE
IMPLIED WARRANTIES OF MERCHANTABILITY AND FITNESS FOR A PARTICULAR PURPOSE
ARE DISCLAIMED. IN NO EVENT SHALL THE REGENTS OR CONTRIBUTORS BE LIABLE FOR
ANY DIRECT, INDIRECT, INCIDENTAL, SPECIAL, EXEMPLARY, OR CONSEQUENTIAL
DAMAGES (INCLUDING, BUT NOT LIMITED TO, PROCUREMENT OF SUBSTITUTE GOODS OR
SERVICES; LOSS OF USE, DATA, OR PROFITS; OR BUSINESS INTERRUPTION) HOWEVER
CAUSED AND ON ANY THEORY OF LIABILITY, WHETHER IN CONTRACT, STRICT
LIABILITY, OR TORT (INCLUDING NEGLIGENCE OR OTHERWISE) ARISING IN ANY WAY
OUT OF THE USE OF THIS SOFTWARE, EVEN IF ADVISED OF THE POSSIBILITY OF SUCH
DAMAGE.
\end{Verbatim}


\section{Release Notes}
\label{changes:release-notes}\label{changes::doc}

\subsection{Version 1.0, 2014-09-05}
\label{changes:version-1-0-2014-09-05}\begin{itemize}
\item {} 
Fix when overwritten project has a git repository

\item {} 
Documentation updates

\item {} 
License section in Sphinx

\item {} 
Django project support with --with-django flag

\item {} 
Travis project support with --with-travis flag

\item {} 
Replaced sh with own implementation

\item {} 
Fix: new \emph{git describe} version to PEP440 conversion

\item {} 
conf.py improvements

\item {} 
Added source code documentation

\item {} 
Fix: Some Python 2/3 compatibility issues

\item {} 
Support for Windows

\item {} 
Dropped Python 2.6 support

\item {} 
Some classifier updates

\end{itemize}


\subsection{Version 0.9, 2014-07-27}
\label{changes:version-0-9-2014-07-27}\begin{itemize}
\item {} 
Documentation updates due to RTD

\item {} 
Added a --force flag

\item {} 
Some cleanups in setup.py

\end{itemize}


\subsection{Version 0.8, 2014-07-25}
\label{changes:version-0-8-2014-07-25}\begin{itemize}
\item {} 
Update to Versioneer 0.10

\item {} 
Moved sphinx-apidoc from setup.py to conf.py

\item {} 
Better support for \emph{make html}

\end{itemize}


\subsection{Version 0.7, 2014-06-05}
\label{changes:version-0-7-2014-06-05}\begin{itemize}
\item {} 
Added Python 3.4 tests and support

\item {} 
Flag --update updates only some files now

\item {} 
Usage of setup\_requires instead of six code

\end{itemize}


\subsection{Version 0.6.1, 2014-05-15}
\label{changes:version-0-6-1-2014-05-15}\begin{itemize}
\item {} 
Fix: Removed six dependency in setup.py

\end{itemize}


\subsection{Version 0.6, 2014-05-14}
\label{changes:version-0-6-2014-05-14}\begin{itemize}
\item {} 
Better usage of six

\item {} 
Return non-zero exit status when doctests fail

\item {} 
Updated README

\item {} 
Fixes in Sphinx Makefile

\end{itemize}


\subsection{Version 0.5, 2014-05-02}
\label{changes:version-0-5-2014-05-02}\begin{itemize}
\item {} 
Simplified some Travis tests

\item {} 
Nicer output in case of errors

\item {} 
Updated PyScaffold's own setup.py

\item {} 
Added --junit\_xml and --coverage\_xml/html option

\item {} 
Updated .gitignore file

\end{itemize}


\subsection{Version 0.4.1, 2014-04-27}
\label{changes:version-0-4-1-2014-04-27}\begin{itemize}
\item {} 
Problem fixed with pytest-cov installation

\end{itemize}


\subsection{Version 0.4, 2014-04-23}
\label{changes:version-0-4-2014-04-23}\begin{itemize}
\item {} 
PEP8 and PyFlakes fixes

\item {} 
Added --version flag

\item {} 
Small fixes and cleanups

\end{itemize}


\subsection{Version 0.3, 2014-04-18}
\label{changes:version-0-3-2014-04-18}\begin{itemize}
\item {} 
PEP8 fixes

\item {} 
More documentation

\item {} 
Added update feature

\item {} 
Fixes in setup.py

\end{itemize}


\subsection{Version 0.2, 2014-04-15}
\label{changes:version-0-2-2014-04-15}\begin{itemize}
\item {} 
Checks when creating the project

\item {} 
Fixes in COPYING

\item {} 
Usage of sh instead of GitPython

\item {} 
PEP8 fixes

\item {} 
Python 3 compatibility

\item {} 
Coverage with Coverall.io

\item {} 
Some more unittests

\end{itemize}


\subsection{Version 0.1.2, 2014-04-10}
\label{changes:version-0-1-2-2014-04-10}\begin{itemize}
\item {} 
Bugfix in Manifest.in

\item {} 
Python 2.6 problems fixed

\end{itemize}


\subsection{Version 0.1.1, 2014-04-10}
\label{changes:version-0-1-1-2014-04-10}\begin{itemize}
\item {} 
Unittesting with Travis

\item {} 
Switch to string.Template

\item {} 
Minor bugfixes

\end{itemize}


\subsection{Version 0.1, 2014-04-03}
\label{changes:version-0-1-2014-04-03}\begin{itemize}
\item {} 
First release

\end{itemize}


\section{pyscaffold}
\label{_rst/modules:pyscaffold}\label{_rst/modules::doc}

\subsection{pyscaffold package}
\label{_rst/pyscaffold::doc}\label{_rst/pyscaffold:pyscaffold-package}

\subsubsection{Submodules}
\label{_rst/pyscaffold:submodules}

\subsubsection{pyscaffold.info module}
\label{_rst/pyscaffold:module-pyscaffold.info}\label{_rst/pyscaffold:pyscaffold-info-module}\index{pyscaffold.info (module)}
Provide general information about the system, user etc.
\index{email() (in module pyscaffold.info)}

\begin{fulllineitems}
\phantomsection\label{_rst/pyscaffold:pyscaffold.info.email}\pysiglinewithargsret{\code{pyscaffold.info.}\bfcode{email}}{}{}
Retrieve the user's email
\begin{quote}\begin{description}
\item[{Returns}] \leavevmode
user's email as string

\end{description}\end{quote}

\end{fulllineitems}

\index{is\_git\_installed() (in module pyscaffold.info)}

\begin{fulllineitems}
\phantomsection\label{_rst/pyscaffold:pyscaffold.info.is_git_installed}\pysiglinewithargsret{\code{pyscaffold.info.}\bfcode{is\_git\_installed}}{}{}
Check if git is installed
\begin{quote}\begin{description}
\item[{Returns}] \leavevmode
boolean

\end{description}\end{quote}

\end{fulllineitems}

\index{project() (in module pyscaffold.info)}

\begin{fulllineitems}
\phantomsection\label{_rst/pyscaffold:pyscaffold.info.project}\pysiglinewithargsret{\code{pyscaffold.info.}\bfcode{project}}{\emph{args}}{}
Update user settings with the settings of an existing PyScaffold project
\begin{quote}\begin{description}
\item[{Parameters}] \leavevmode
\textbf{args} -- command line parameters as \href{http://docs.python.org/2.7/library/argparse.html\#argparse.Namespace}{\code{argparse.Namespace}}

\item[{Returns}] \leavevmode
updated command line parameters as \href{http://docs.python.org/2.7/library/argparse.html\#argparse.Namespace}{\code{argparse.Namespace}}

\end{description}\end{quote}

\end{fulllineitems}

\index{username() (in module pyscaffold.info)}

\begin{fulllineitems}
\phantomsection\label{_rst/pyscaffold:pyscaffold.info.username}\pysiglinewithargsret{\code{pyscaffold.info.}\bfcode{username}}{}{}
Retrieve the user's name
\begin{quote}\begin{description}
\item[{Returns}] \leavevmode
user's name as string

\end{description}\end{quote}

\end{fulllineitems}



\subsubsection{pyscaffold.repo module}
\label{_rst/pyscaffold:module-pyscaffold.repo}\label{_rst/pyscaffold:pyscaffold-repo-module}\index{pyscaffold.repo (module)}
Functionality for working with a git repository
\index{git\_tree\_add() (in module pyscaffold.repo)}

\begin{fulllineitems}
\phantomsection\label{_rst/pyscaffold:pyscaffold.repo.git_tree_add}\pysiglinewithargsret{\code{pyscaffold.repo.}\bfcode{git\_tree\_add}}{\emph{struct}, \emph{prefix='`}}{}
Adds recursively a directory structure to git
\begin{quote}\begin{description}
\item[{Parameters}] \leavevmode\begin{itemize}
\item {} 
\textbf{struct} -- directory structure as dictionary of dictionaries

\item {} 
\textbf{prefix} -- prefix for the given directory structure as string

\end{itemize}

\end{description}\end{quote}

\end{fulllineitems}

\index{init\_commit\_repo() (in module pyscaffold.repo)}

\begin{fulllineitems}
\phantomsection\label{_rst/pyscaffold:pyscaffold.repo.init_commit_repo}\pysiglinewithargsret{\code{pyscaffold.repo.}\bfcode{init\_commit\_repo}}{\emph{project}, \emph{struct}}{}
Initialize a git repository
\begin{quote}\begin{description}
\item[{Parameters}] \leavevmode\begin{itemize}
\item {} 
\textbf{project} -- path to the project as string

\item {} 
\textbf{struct} -- directory structure as dictionary of dictionaries

\end{itemize}

\end{description}\end{quote}

\end{fulllineitems}

\index{is\_git\_repo() (in module pyscaffold.repo)}

\begin{fulllineitems}
\phantomsection\label{_rst/pyscaffold:pyscaffold.repo.is_git_repo}\pysiglinewithargsret{\code{pyscaffold.repo.}\bfcode{is\_git\_repo}}{\emph{folder}}{}
Check if a folder is a git repository
\begin{quote}\begin{description}
\item[{Parameters}] \leavevmode
\textbf{folder} -- path as string

\end{description}\end{quote}

\end{fulllineitems}



\subsubsection{pyscaffold.runner module}
\label{_rst/pyscaffold:module-pyscaffold.runner}\label{_rst/pyscaffold:pyscaffold-runner-module}\index{pyscaffold.runner (module)}
Command-Line-Interface of PyScaffold
\index{main() (in module pyscaffold.runner)}

\begin{fulllineitems}
\phantomsection\label{_rst/pyscaffold:pyscaffold.runner.main}\pysiglinewithargsret{\code{pyscaffold.runner.}\bfcode{main}}{\emph{args}}{}
Main entry point of PyScaffold
\begin{quote}\begin{description}
\item[{Parameters}] \leavevmode
\textbf{args} -- command line parameters as list of strings

\end{description}\end{quote}

\end{fulllineitems}

\index{parse\_args() (in module pyscaffold.runner)}

\begin{fulllineitems}
\phantomsection\label{_rst/pyscaffold:pyscaffold.runner.parse_args}\pysiglinewithargsret{\code{pyscaffold.runner.}\bfcode{parse\_args}}{\emph{args}}{}
Command line parameters
\begin{quote}\begin{description}
\item[{Parameters}] \leavevmode
\textbf{args} -- command line parameters as list of strings

\item[{Returns}] \leavevmode
command line parameters as \href{http://docs.python.org/2.7/library/argparse.html\#argparse.Namespace}{\code{argparse.Namespace}}

\end{description}\end{quote}

\end{fulllineitems}

\index{run() (in module pyscaffold.runner)}

\begin{fulllineitems}
\phantomsection\label{_rst/pyscaffold:pyscaffold.runner.run}\pysiglinewithargsret{\code{pyscaffold.runner.}\bfcode{run}}{\emph{*args}, \emph{**kwargs}}{}
Entry point for setup.py

\end{fulllineitems}



\subsubsection{pyscaffold.shell module}
\label{_rst/pyscaffold:pyscaffold-shell-module}\label{_rst/pyscaffold:module-pyscaffold.shell}\index{pyscaffold.shell (module)}
Shell commands like git, django-admin.py etc.
\index{Command (class in pyscaffold.shell)}

\begin{fulllineitems}
\phantomsection\label{_rst/pyscaffold:pyscaffold.shell.Command}\pysiglinewithargsret{\strong{class }\code{pyscaffold.shell.}\bfcode{Command}}{\emph{command}}{}
Bases: \code{object}

Shell command that can be called with flags like git(`add', `file')
\begin{quote}\begin{description}
\item[{Parameters}] \leavevmode
\textbf{command} -- command to handle

\end{description}\end{quote}

\end{fulllineitems}

\index{called\_process\_error2exit\_decorator() (in module pyscaffold.shell)}

\begin{fulllineitems}
\phantomsection\label{_rst/pyscaffold:pyscaffold.shell.called_process_error2exit_decorator}\pysiglinewithargsret{\code{pyscaffold.shell.}\bfcode{called\_process\_error2exit\_decorator}}{\emph{func}}{}
Decorator to convert given CalledProcessError to an exit message

This avoids displaying nasty stack traces to end-users

\end{fulllineitems}

\index{django\_admin (in module pyscaffold.shell)}

\begin{fulllineitems}
\phantomsection\label{_rst/pyscaffold:pyscaffold.shell.django_admin}\pysigline{\code{pyscaffold.shell.}\bfcode{django\_admin}\strong{ = \textless{}pyscaffold.shell.Command object at 0x4e3e490\textgreater{}}}
Command for django-admin.py

\end{fulllineitems}

\index{git (in module pyscaffold.shell)}

\begin{fulllineitems}
\phantomsection\label{_rst/pyscaffold:pyscaffold.shell.git}\pysigline{\code{pyscaffold.shell.}\bfcode{git}\strong{ = \textless{}pyscaffold.shell.Command object at 0x4e3e450\textgreater{}}}
Command for git

\end{fulllineitems}



\subsubsection{pyscaffold.structure module}
\label{_rst/pyscaffold:module-pyscaffold.structure}\label{_rst/pyscaffold:pyscaffold-structure-module}\index{pyscaffold.structure (module)}
Functionality to generate and work with the directory structure of a project
\index{create\_django\_proj() (in module pyscaffold.structure)}

\begin{fulllineitems}
\phantomsection\label{_rst/pyscaffold:pyscaffold.structure.create_django_proj}\pysiglinewithargsret{\code{pyscaffold.structure.}\bfcode{create\_django\_proj}}{\emph{args}}{}
Creates a standard Django project with django-admin.py
\begin{quote}\begin{description}
\item[{Parameters}] \leavevmode
\textbf{args} -- command line parameters as \href{http://docs.python.org/2.7/library/argparse.html\#argparse.Namespace}{\code{argparse.Namespace}}

\end{description}\end{quote}

\end{fulllineitems}

\index{create\_structure() (in module pyscaffold.structure)}

\begin{fulllineitems}
\phantomsection\label{_rst/pyscaffold:pyscaffold.structure.create_structure}\pysiglinewithargsret{\code{pyscaffold.structure.}\bfcode{create\_structure}}{\emph{struct}, \emph{prefix=None}, \emph{update=False}}{}
Manifests a directory structure in the filesystem
\begin{quote}\begin{description}
\item[{Parameters}] \leavevmode\begin{itemize}
\item {} 
\textbf{struct} -- directory structure as dictionary of dictionaries

\item {} 
\textbf{prefix} -- prefix path for the structure

\item {} 
\textbf{update} -- update an existing directory structure as boolean

\end{itemize}

\end{description}\end{quote}

\end{fulllineitems}

\index{make\_structure() (in module pyscaffold.structure)}

\begin{fulllineitems}
\phantomsection\label{_rst/pyscaffold:pyscaffold.structure.make_structure}\pysiglinewithargsret{\code{pyscaffold.structure.}\bfcode{make\_structure}}{\emph{args}}{}
Creates the project structure as dictionary of dictionaries
\begin{quote}\begin{description}
\item[{Parameters}] \leavevmode
\textbf{args} -- command line parameters as \href{http://docs.python.org/2.7/library/argparse.html\#argparse.Namespace}{\code{argparse.Namespace}}

\item[{Returns}] \leavevmode
structure as dictionary of dictionaries

\end{description}\end{quote}

\end{fulllineitems}

\index{set\_default\_args() (in module pyscaffold.structure)}

\begin{fulllineitems}
\phantomsection\label{_rst/pyscaffold:pyscaffold.structure.set_default_args}\pysiglinewithargsret{\code{pyscaffold.structure.}\bfcode{set\_default\_args}}{\emph{args}}{}
Set default arguments for some parameters
\begin{quote}\begin{description}
\item[{Parameters}] \leavevmode
\textbf{args} -- command line parameters as \href{http://docs.python.org/2.7/library/argparse.html\#argparse.Namespace}{\code{argparse.Namespace}}

\item[{Returns}] \leavevmode
command line parameters as \href{http://docs.python.org/2.7/library/argparse.html\#argparse.Namespace}{\code{argparse.Namespace}}

\end{description}\end{quote}

\end{fulllineitems}



\subsubsection{pyscaffold.templates module}
\label{_rst/pyscaffold:pyscaffold-templates-module}\label{_rst/pyscaffold:module-pyscaffold.templates}\index{pyscaffold.templates (module)}
Templates for all files of the project's scaffold
\index{authors() (in module pyscaffold.templates)}

\begin{fulllineitems}
\phantomsection\label{_rst/pyscaffold:pyscaffold.templates.authors}\pysiglinewithargsret{\code{pyscaffold.templates.}\bfcode{authors}}{\emph{args}}{}
Template of AUTHORS.rst
\begin{quote}\begin{description}
\item[{Parameters}] \leavevmode
\textbf{args} -- command line parameters as \href{http://docs.python.org/2.7/library/argparse.html\#argparse.Namespace}{\code{argparse.Namespace}}

\item[{Returns}] \leavevmode
file content as string

\end{description}\end{quote}

\end{fulllineitems}

\index{copying() (in module pyscaffold.templates)}

\begin{fulllineitems}
\phantomsection\label{_rst/pyscaffold:pyscaffold.templates.copying}\pysiglinewithargsret{\code{pyscaffold.templates.}\bfcode{copying}}{\emph{args}}{}
Template of COPYING
\begin{quote}\begin{description}
\item[{Parameters}] \leavevmode
\textbf{args} -- command line parameters as \href{http://docs.python.org/2.7/library/argparse.html\#argparse.Namespace}{\code{argparse.Namespace}}

\item[{Returns}] \leavevmode
file content as string

\end{description}\end{quote}

\end{fulllineitems}

\index{coveragerc() (in module pyscaffold.templates)}

\begin{fulllineitems}
\phantomsection\label{_rst/pyscaffold:pyscaffold.templates.coveragerc}\pysiglinewithargsret{\code{pyscaffold.templates.}\bfcode{coveragerc}}{\emph{args}}{}
Template of .coveragerc
\begin{quote}\begin{description}
\item[{Parameters}] \leavevmode
\textbf{args} -- command line parameters as \href{http://docs.python.org/2.7/library/argparse.html\#argparse.Namespace}{\code{argparse.Namespace}}

\item[{Returns}] \leavevmode
file content as string

\end{description}\end{quote}

\end{fulllineitems}

\index{get\_template() (in module pyscaffold.templates)}

\begin{fulllineitems}
\phantomsection\label{_rst/pyscaffold:pyscaffold.templates.get_template}\pysiglinewithargsret{\code{pyscaffold.templates.}\bfcode{get\_template}}{\emph{name}}{}
Retrieve the template by name
\begin{quote}\begin{description}
\item[{Parameters}] \leavevmode
\textbf{name} -- name of template

\item[{Returns}] \leavevmode
template as \href{http://docs.python.org/2.7/library/string.html\#string.Template}{\code{string.Template}}

\end{description}\end{quote}

\end{fulllineitems}

\index{gitignore() (in module pyscaffold.templates)}

\begin{fulllineitems}
\phantomsection\label{_rst/pyscaffold:pyscaffold.templates.gitignore}\pysiglinewithargsret{\code{pyscaffold.templates.}\bfcode{gitignore}}{\emph{args}}{}
Template of .gitignore
\begin{quote}\begin{description}
\item[{Parameters}] \leavevmode
\textbf{args} -- command line parameters as \href{http://docs.python.org/2.7/library/argparse.html\#argparse.Namespace}{\code{argparse.Namespace}}

\item[{Returns}] \leavevmode
file content as string

\end{description}\end{quote}

\end{fulllineitems}

\index{gitignore\_empty() (in module pyscaffold.templates)}

\begin{fulllineitems}
\phantomsection\label{_rst/pyscaffold:pyscaffold.templates.gitignore_empty}\pysiglinewithargsret{\code{pyscaffold.templates.}\bfcode{gitignore\_empty}}{\emph{args}}{}
Template of empty .gitignore
\begin{quote}\begin{description}
\item[{Parameters}] \leavevmode
\textbf{args} -- command line parameters as \href{http://docs.python.org/2.7/library/argparse.html\#argparse.Namespace}{\code{argparse.Namespace}}

\item[{Returns}] \leavevmode
file content as string

\end{description}\end{quote}

\end{fulllineitems}

\index{init() (in module pyscaffold.templates)}

\begin{fulllineitems}
\phantomsection\label{_rst/pyscaffold:pyscaffold.templates.init}\pysiglinewithargsret{\code{pyscaffold.templates.}\bfcode{init}}{\emph{args}}{}
Template of \_\_init\_\_.py
\begin{quote}\begin{description}
\item[{Parameters}] \leavevmode
\textbf{args} -- command line parameters as \href{http://docs.python.org/2.7/library/argparse.html\#argparse.Namespace}{\code{argparse.Namespace}}

\item[{Returns}] \leavevmode
file content as string

\end{description}\end{quote}

\end{fulllineitems}

\index{manifest\_in() (in module pyscaffold.templates)}

\begin{fulllineitems}
\phantomsection\label{_rst/pyscaffold:pyscaffold.templates.manifest_in}\pysiglinewithargsret{\code{pyscaffold.templates.}\bfcode{manifest\_in}}{\emph{args}}{}
Template of MANIFEST.in
\begin{quote}\begin{description}
\item[{Parameters}] \leavevmode
\textbf{args} -- command line parameters as \href{http://docs.python.org/2.7/library/argparse.html\#argparse.Namespace}{\code{argparse.Namespace}}

\item[{Returns}] \leavevmode
file content as string

\end{description}\end{quote}

\end{fulllineitems}

\index{readme() (in module pyscaffold.templates)}

\begin{fulllineitems}
\phantomsection\label{_rst/pyscaffold:pyscaffold.templates.readme}\pysiglinewithargsret{\code{pyscaffold.templates.}\bfcode{readme}}{\emph{args}}{}
Template of README.rst
\begin{quote}\begin{description}
\item[{Parameters}] \leavevmode
\textbf{args} -- command line parameters as \href{http://docs.python.org/2.7/library/argparse.html\#argparse.Namespace}{\code{argparse.Namespace}}

\item[{Returns}] \leavevmode
file content as string

\end{description}\end{quote}

\end{fulllineitems}

\index{requirements() (in module pyscaffold.templates)}

\begin{fulllineitems}
\phantomsection\label{_rst/pyscaffold:pyscaffold.templates.requirements}\pysiglinewithargsret{\code{pyscaffold.templates.}\bfcode{requirements}}{\emph{args}}{}
Template of requirements.txt
\begin{quote}\begin{description}
\item[{Parameters}] \leavevmode
\textbf{args} -- command line parameters as \href{http://docs.python.org/2.7/library/argparse.html\#argparse.Namespace}{\code{argparse.Namespace}}

\item[{Returns}] \leavevmode
file content as string

\end{description}\end{quote}

\end{fulllineitems}

\index{setup() (in module pyscaffold.templates)}

\begin{fulllineitems}
\phantomsection\label{_rst/pyscaffold:pyscaffold.templates.setup}\pysiglinewithargsret{\code{pyscaffold.templates.}\bfcode{setup}}{\emph{args}}{}
Template of setup.py
\begin{quote}\begin{description}
\item[{Parameters}] \leavevmode
\textbf{args} -- command line parameters as \href{http://docs.python.org/2.7/library/argparse.html\#argparse.Namespace}{\code{argparse.Namespace}}

\item[{Returns}] \leavevmode
file content as string

\end{description}\end{quote}

\end{fulllineitems}

\index{sphinx\_conf() (in module pyscaffold.templates)}

\begin{fulllineitems}
\phantomsection\label{_rst/pyscaffold:pyscaffold.templates.sphinx_conf}\pysiglinewithargsret{\code{pyscaffold.templates.}\bfcode{sphinx\_conf}}{\emph{args}}{}
Template of conf.py
\begin{quote}\begin{description}
\item[{Parameters}] \leavevmode
\textbf{args} -- command line parameters as \href{http://docs.python.org/2.7/library/argparse.html\#argparse.Namespace}{\code{argparse.Namespace}}

\item[{Returns}] \leavevmode
file content as string

\end{description}\end{quote}

\end{fulllineitems}

\index{sphinx\_index() (in module pyscaffold.templates)}

\begin{fulllineitems}
\phantomsection\label{_rst/pyscaffold:pyscaffold.templates.sphinx_index}\pysiglinewithargsret{\code{pyscaffold.templates.}\bfcode{sphinx\_index}}{\emph{args}}{}
Template of index.rst
\begin{quote}\begin{description}
\item[{Parameters}] \leavevmode
\textbf{args} -- command line parameters as \href{http://docs.python.org/2.7/library/argparse.html\#argparse.Namespace}{\code{argparse.Namespace}}

\item[{Returns}] \leavevmode
file content as string

\end{description}\end{quote}

\end{fulllineitems}

\index{sphinx\_license() (in module pyscaffold.templates)}

\begin{fulllineitems}
\phantomsection\label{_rst/pyscaffold:pyscaffold.templates.sphinx_license}\pysiglinewithargsret{\code{pyscaffold.templates.}\bfcode{sphinx\_license}}{\emph{args}}{}
Template of license.rst
\begin{quote}\begin{description}
\item[{Parameters}] \leavevmode
\textbf{args} -- command line parameters as \href{http://docs.python.org/2.7/library/argparse.html\#argparse.Namespace}{\code{argparse.Namespace}}

\item[{Returns}] \leavevmode
file content as string

\end{description}\end{quote}

\end{fulllineitems}

\index{sphinx\_makefile() (in module pyscaffold.templates)}

\begin{fulllineitems}
\phantomsection\label{_rst/pyscaffold:pyscaffold.templates.sphinx_makefile}\pysiglinewithargsret{\code{pyscaffold.templates.}\bfcode{sphinx\_makefile}}{\emph{args}}{}
Template of Sphinx's Makefile
\begin{quote}\begin{description}
\item[{Parameters}] \leavevmode
\textbf{args} -- command line parameters as \href{http://docs.python.org/2.7/library/argparse.html\#argparse.Namespace}{\code{argparse.Namespace}}

\item[{Returns}] \leavevmode
file content as string

\end{description}\end{quote}

\end{fulllineitems}

\index{travis() (in module pyscaffold.templates)}

\begin{fulllineitems}
\phantomsection\label{_rst/pyscaffold:pyscaffold.templates.travis}\pysiglinewithargsret{\code{pyscaffold.templates.}\bfcode{travis}}{\emph{args}}{}
Template of .travis.yml
\begin{quote}\begin{description}
\item[{Parameters}] \leavevmode
\textbf{args} -- command line parameters as \href{http://docs.python.org/2.7/library/argparse.html\#argparse.Namespace}{\code{argparse.Namespace}}

\item[{Returns}] \leavevmode
file content as string

\end{description}\end{quote}

\end{fulllineitems}

\index{travis\_install() (in module pyscaffold.templates)}

\begin{fulllineitems}
\phantomsection\label{_rst/pyscaffold:pyscaffold.templates.travis_install}\pysiglinewithargsret{\code{pyscaffold.templates.}\bfcode{travis\_install}}{\emph{args}}{}
Template of travis\_install.sh
\begin{quote}\begin{description}
\item[{Parameters}] \leavevmode
\textbf{args} -- command line parameters as \href{http://docs.python.org/2.7/library/argparse.html\#argparse.Namespace}{\code{argparse.Namespace}}

\item[{Returns}] \leavevmode
file content as string

\end{description}\end{quote}

\end{fulllineitems}

\index{version() (in module pyscaffold.templates)}

\begin{fulllineitems}
\phantomsection\label{_rst/pyscaffold:pyscaffold.templates.version}\pysiglinewithargsret{\code{pyscaffold.templates.}\bfcode{version}}{\emph{args}}{}
Template of \_version.py
\begin{quote}\begin{description}
\item[{Parameters}] \leavevmode
\textbf{args} -- command line parameters as \href{http://docs.python.org/2.7/library/argparse.html\#argparse.Namespace}{\code{argparse.Namespace}}

\item[{Returns}] \leavevmode
file content as string

\end{description}\end{quote}

\end{fulllineitems}

\index{versioneer() (in module pyscaffold.templates)}

\begin{fulllineitems}
\phantomsection\label{_rst/pyscaffold:pyscaffold.templates.versioneer}\pysiglinewithargsret{\code{pyscaffold.templates.}\bfcode{versioneer}}{\emph{args}}{}
Template of versioneer.py
\begin{quote}\begin{description}
\item[{Parameters}] \leavevmode
\textbf{args} -- command line parameters as \href{http://docs.python.org/2.7/library/argparse.html\#argparse.Namespace}{\code{argparse.Namespace}}

\item[{Returns}] \leavevmode
file content as string

\end{description}\end{quote}

\end{fulllineitems}



\subsubsection{pyscaffold.utils module}
\label{_rst/pyscaffold:pyscaffold-utils-module}\label{_rst/pyscaffold:module-pyscaffold.utils}\index{pyscaffold.utils (module)}
Miscellaneous utilities and tools
\index{ObjKeeper (class in pyscaffold.utils)}

\begin{fulllineitems}
\phantomsection\label{_rst/pyscaffold:pyscaffold.utils.ObjKeeper}\pysiglinewithargsret{\strong{class }\code{pyscaffold.utils.}\bfcode{ObjKeeper}}{\emph{name}, \emph{bases}, \emph{dct}}{}
Bases: \code{type}

Metaclass to keep track of generated instances of a class
\index{instances (pyscaffold.utils.ObjKeeper attribute)}

\begin{fulllineitems}
\phantomsection\label{_rst/pyscaffold:pyscaffold.utils.ObjKeeper.instances}\pysigline{\bfcode{instances}\strong{ = \{\}}}
\end{fulllineitems}


\end{fulllineitems}

\index{capture\_objs() (in module pyscaffold.utils)}

\begin{fulllineitems}
\phantomsection\label{_rst/pyscaffold:pyscaffold.utils.capture_objs}\pysiglinewithargsret{\code{pyscaffold.utils.}\bfcode{capture\_objs}}{\emph{cls}}{}
Captures the instances of a given class during runtime
\begin{quote}
\begin{quote}\begin{description}
\item[{param cls}] \leavevmode
class to capture

\item[{return}] \leavevmode
dynamic list with references to all instances of \code{cls}

\end{description}\end{quote}
\end{quote}

\end{fulllineitems}

\index{chdir() (in module pyscaffold.utils)}

\begin{fulllineitems}
\phantomsection\label{_rst/pyscaffold:pyscaffold.utils.chdir}\pysiglinewithargsret{\code{pyscaffold.utils.}\bfcode{chdir}}{\emph{*args}, \emph{**kwds}}{}
Contextmanager to change into a directory
\begin{quote}\begin{description}
\item[{Parameters}] \leavevmode
\textbf{path} -- path to change into as string

\end{description}\end{quote}

\end{fulllineitems}

\index{exceptions2exit() (in module pyscaffold.utils)}

\begin{fulllineitems}
\phantomsection\label{_rst/pyscaffold:pyscaffold.utils.exceptions2exit}\pysiglinewithargsret{\code{pyscaffold.utils.}\bfcode{exceptions2exit}}{\emph{exception\_list}}{}
Decorator to convert given exceptions to exit messages

This avoids displaying nasty stack traces to end-users
\begin{quote}\begin{description}
\item[{Parameters}] \leavevmode
\textbf{exception\_list} -- list of exceptions to convert

\end{description}\end{quote}

\end{fulllineitems}

\index{git2pep440() (in module pyscaffold.utils)}

\begin{fulllineitems}
\phantomsection\label{_rst/pyscaffold:pyscaffold.utils.git2pep440}\pysiglinewithargsret{\code{pyscaffold.utils.}\bfcode{git2pep440}}{\emph{ver\_str}}{}
Converts a git description to a PEP440 conforming string
\begin{quote}\begin{description}
\item[{Parameters}] \leavevmode
\textbf{ver\_str} -- git version description

\item[{Returns}] \leavevmode
PEP440 version description

\end{description}\end{quote}

\end{fulllineitems}

\index{is\_valid\_identifier() (in module pyscaffold.utils)}

\begin{fulllineitems}
\phantomsection\label{_rst/pyscaffold:pyscaffold.utils.is_valid_identifier}\pysiglinewithargsret{\code{pyscaffold.utils.}\bfcode{is\_valid\_identifier}}{\emph{string}}{}
Check if string is a valid package name
\begin{quote}\begin{description}
\item[{Parameters}] \leavevmode
\textbf{string} -- package name as string

\item[{Returns}] \leavevmode
boolean

\end{description}\end{quote}

\end{fulllineitems}

\index{list2str() (in module pyscaffold.utils)}

\begin{fulllineitems}
\phantomsection\label{_rst/pyscaffold:pyscaffold.utils.list2str}\pysiglinewithargsret{\code{pyscaffold.utils.}\bfcode{list2str}}{\emph{lst}, \emph{indent=0}}{}
Generate a Python syntax list string with an indention
\begin{quote}\begin{description}
\item[{Parameters}] \leavevmode\begin{itemize}
\item {} 
\textbf{lst} -- list

\item {} 
\textbf{indent} -- indention as integer

\end{itemize}

\item[{Returns}] \leavevmode
string

\end{description}\end{quote}

\end{fulllineitems}

\index{make\_valid\_identifier() (in module pyscaffold.utils)}

\begin{fulllineitems}
\phantomsection\label{_rst/pyscaffold:pyscaffold.utils.make_valid_identifier}\pysiglinewithargsret{\code{pyscaffold.utils.}\bfcode{make\_valid\_identifier}}{\emph{string}}{}
Try to make a valid package name identifier from a string
\begin{quote}\begin{description}
\item[{Parameters}] \leavevmode
\textbf{string} -- invalid package name as string

\item[{Returns}] \leavevmode
valid package name as string or \code{RuntimeError}

\end{description}\end{quote}

\end{fulllineitems}

\index{safe\_get() (in module pyscaffold.utils)}

\begin{fulllineitems}
\phantomsection\label{_rst/pyscaffold:pyscaffold.utils.safe_get}\pysiglinewithargsret{\code{pyscaffold.utils.}\bfcode{safe\_get}}{\emph{namespace}, \emph{attr}}{}
Safely retrieve the value of a namespace's attribute
\begin{quote}\begin{description}
\item[{Parameters}] \leavevmode\begin{itemize}
\item {} 
\textbf{namespace} -- namespace as \href{http://docs.python.org/2.7/library/argparse.html\#argparse.Namespace}{\code{argparse.Namespace}} object

\item {} 
\textbf{attr} -- attribute name as string

\end{itemize}

\item[{Returns}] \leavevmode
value of the attribute or None

\end{description}\end{quote}

\end{fulllineitems}

\index{safe\_set() (in module pyscaffold.utils)}

\begin{fulllineitems}
\phantomsection\label{_rst/pyscaffold:pyscaffold.utils.safe_set}\pysiglinewithargsret{\code{pyscaffold.utils.}\bfcode{safe\_set}}{\emph{namespace}, \emph{attr}, \emph{value}}{}
Safely set an attribute of a namespace object

The new attribute is set only if the attribute did not exist or was None.
\begin{quote}\begin{description}
\item[{Parameters}] \leavevmode\begin{itemize}
\item {} 
\textbf{namespace} -- namespace as \href{http://docs.python.org/2.7/library/argparse.html\#argparse.Namespace}{\code{argparse.Namespace}} object

\item {} 
\textbf{attr} -- attribute name as string

\item {} 
\textbf{value} -- value for new attribute

\end{itemize}

\end{description}\end{quote}

\end{fulllineitems}



\subsubsection{Module contents}
\label{_rst/pyscaffold:module-contents}\label{_rst/pyscaffold:module-pyscaffold}\index{pyscaffold (module)}

\chapter{Indices and tables}
\label{index:indices-and-tables}\begin{itemize}
\item {} 
\emph{genindex}

\item {} 
\emph{modindex}

\item {} 
\emph{search}

\end{itemize}


\renewcommand{\indexname}{Python Module Index}
\begin{theindex}
\def\bigletter#1{{\Large\sffamily#1}\nopagebreak\vspace{1mm}}
\bigletter{p}
\item {\texttt{pyscaffold}}, \pageref{_rst/pyscaffold:module-pyscaffold}
\item {\texttt{pyscaffold.info}}, \pageref{_rst/pyscaffold:module-pyscaffold.info}
\item {\texttt{pyscaffold.repo}}, \pageref{_rst/pyscaffold:module-pyscaffold.repo}
\item {\texttt{pyscaffold.runner}}, \pageref{_rst/pyscaffold:module-pyscaffold.runner}
\item {\texttt{pyscaffold.shell}}, \pageref{_rst/pyscaffold:module-pyscaffold.shell}
\item {\texttt{pyscaffold.structure}}, \pageref{_rst/pyscaffold:module-pyscaffold.structure}
\item {\texttt{pyscaffold.templates}}, \pageref{_rst/pyscaffold:module-pyscaffold.templates}
\item {\texttt{pyscaffold.utils}}, \pageref{_rst/pyscaffold:module-pyscaffold.utils}
\end{theindex}

\renewcommand{\indexname}{Index}
\printindex
\end{document}
